%%%%%%%%%%%%%%%%%%%%%%%%%%%%%%%%%%%%%%%%%
% Medium Length Professional CV
% LaTeX Template
% Version 2.0 (8/5/13)
%
% This template has been downloaded from:
% http://www.LaTeXTemplates.com
%
% Original author:
% Rishi Shah 
%
% Important note:
% This template requires the resume.cls file to be in the same directory as the
% .tex file. The resume.cls file provides the resume style used for structuring the
% document.
%
%%%%%%%%%%%%%%%%%%%%%%%%%%%%%%%%%%%%%%%%%


%----------------------------------------------------------------------------------------
%	PACKAGES AND OTHER DOCUMENT CONFIGURATIONS
%----------------------------------------------------------------------------------------

\documentclass{resume} % Use the custom resume.cls style

\usepackage{hyperref}

\usepackage[left=0.75in,top=0.6in,right=0.75in,bottom=0.6in]{geometry} % Document margins
\newcommand{\tab}[1]{\hspace{.2667\textwidth}\rlap{#1}}
\newcommand{\itab}[1]{\hspace{0em}\rlap{#1}}
\name{Pietro Carrara} % Your name
%\address{156 Kasturi, Balajinagar, Sangli 416416} % Your address
%\address{123 Pleasant Lane \\ City, State 12345} % Your secondary addess (optional)
\address{\href{tel:(51)99252-7334}{(51)99252-7334} \\ \href{mailto:pietro.carrara@inf.ufrgs.br}{pietro.carrara@inf.ufrgs.br}} % Your phone number and email
\address{\href{https://github.com/PietroCarrara}{github.com/PietroCarrara}}

\begin{document}

%----------------------------------------------------------------------------------------
%	EDUCATION SECTION
%----------------------------------------------------------------------------------------

\begin{rSection}{Educação}

{\bf Instituto Federal de Educação, Ciência e Tecnologia do Rio Grande do Sul -- Campus Canoas} \hfill {\em 2015 -- 2018} 
\\ Graduado como Técnico em Informática

{\bf Universidade Federal do Rio Grande do Sul} \hfill {\em Agosto 2019 -- Presente} 
\\ Cursando Ciência da Computação

\end{rSection}

\begin{rSection}{Objetivos}
Trabalhar em uma organização que ofereça oportunidades de crescimento pessoal e profissional, através do aprendizado de
novas tecnologias ou da resolução problemas desafiadores.

Buscar certo nível de liberdade como desenvolvedor, para analizar e opinar sobre soluções apresentadas, assim como
discutir e propor outas soluções em conjunto com o time.
\end{rSection}

%--------------------------------------------------------------------------------
%    Projects And Seminars
%-----------------------------------------------------------------------------------------------
\begin{rSection}{Projetos}
{\bf Code Overlord: Jogo Para Auxílio do Ensido de Programação Orientada a Objetos}\\
Projeto desenvolvido como trabalho de conclusão de curso para obtenção do grau de técnico em informática pelo IFRS --
Canoas. O projeto é um jogo desenvolvido em C\#, Javascript, Lua e Go, onde o jogador encontrará diversos problemas e deverá resolvê-los aplicando conceitos da programação orientada a objeto.
\end{rSection}
%----------------------------------------------------------------------------------------
%	TECHNICAL STRENGTHS SECTION
%----------------------------------------------------------------------------------------

% \begin{rSection}{Conhecimentos}

% \begin{tabular}{ @{} >{\bfseries}l @{\hspace{6ex}} l }
% Linguagens \ & Javascript, PHP, SQL, C\#, Go \\
% Frameworks e Ferramentas & Laravel, Linux, JQuery \\
% \end{tabular}

% \end{rSection}

%----------------------------------------------------------------------------------------
%	WORK EXPERIENCE SECTION
%----------------------------------------------------------------------------------------

\begin{rSection}{Experiência de Trabalho}

\begin{rSubsection}{WPlay - Marketing Interativo}{Março 2019}{Programador de Sistemas}{Novo Hamburgo}
\item Desenvolvimento e manutenção de sistemas web.
\item Tecnologias: PHP7, Laravel, MySQL, Vue.js, HTML, CSS, JS
\end{rSubsection}

\begin{rSubsection}{DBSeller Sistemas Integrados}{Abril 2019 -- Julho 2019}{Desenvolvedor Nível 1}{Porto Alegre}
\item Desenvolvimento de novas funcionalidades/manutenção do software aberto de gestão de recursos humanos E-Cidade
\item Tecnologias: PHP5, PostgreSQL, HTML, CSS, JS
\end{rSubsection}

\end{rSection}


\begin{rSection}{Atividades}
 
 Palestra entitulada ``O uso de um carrinho como elemento que estimula o Ensino de Física e Matemática'' no Salão de
 Iniciação Científica e Tecnológica do IFRS no ano de 2016.
 
 Competiu nos anos de 2017 e 2018 na maratona de programação de 12 horas ``CodeRace''.
 
 Contribuições para projetos \textit{Open-Source} como: \href{https://github.com/KDE/krita/commits?author=PietroCarrara}{Krita},
 \href{https://github.com/eonpatapon/mpDris2/commits?author=PietroCarrara}{mpDris2} e
 \href{https://github.com/hellosiyan/Viewnior/commits?author=PietroCarrara}{Viewnior}.
\end{rSection}

\end{document}
